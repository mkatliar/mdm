% mn2esample.tex
%
% v2.1 released 22nd May 2002 (G. Hutton)
%
% The mnsample.tex file has been amended to highlight
% the proper use of LaTeX2e code with the class file
% and using natbib cross-referencing. These changes
% do not reflect the original paper by A. V. Raveendran.
%
% Previous versions of this sample document were
% compatible with the LaTeX 2.09 style file mn.sty
% v1.2 released 5th September 1994 (M. Reed)
% v1.1 released 18th July 1994
% v1.0 released 28th January 1994

\documentclass[useAMS,usenatbib]{mn2e}
\usepackage{dsfont}
\usepackage{amsmath}

\newcommand{\y}{\ensuremath{\mathbf{y}}}
\newcommand{\e}{\ensuremath{\mathbf{e}}}
\newcommand{\z}{\ensuremath{\mathbf{z}}}
\newcommand{\x}{\ensuremath{\mathbf{x}}}
\newcommand{\E}{\ensuremath{\mathrm{E}}}
\newcommand{\X}{\ensuremath{\mathrm{X}}}
\newcommand{\rr}{\ensuremath{\mathrm{r}}}
\newcommand{\A}{\ensuremath{\mathrm{A}}}
\newcommand{\T}{\ensuremath{\mathrm{T}}}
\newcommand{\s}{\ensuremath{\mathrm{s}}}
\newcommand{\F}{\ensuremath{\mathcal{F}}}

% If your system does not have the AMS fonts version 2.0 installed, then
% remove the useAMS option.
%
% useAMS allows you to obtain upright Greek characters.
% e.g. \umu, \upi etc.  See the section on "Upright Greek characters" in
% this guide for further information.
%
% If you are using AMS 2.0 fonts, bold math letters/symbols are available
% at a larger range of sizes for NFSS release 1 and 2 (using \boldmath or
% preferably \bmath).
%
% The usenatbib command allows the use of Patrick Daly's natbib.sty for
% cross-referencing.
%
% If you wish to typeset the paper in Times font (if you do not have the
% PostScript Type 1 Computer Modern fonts you will need to do this to get
% smoother fonts in a PDF file) then uncomment the next line
% \usepackage{Times}

%%%%% AUTHORS - PLACE YOUR OWN MACROS HERE %%%%%


%%%%%%%%%%%%%%%%%%%%%%%%%%%%%%%%%%%%%%%%%%%%%%%%

\title[Visible galaxy shear measurement in the presence of a convolution kernel]{Visible galaxy shear measurement in the presence of a convolution kernel}
\author[Mikhail Katliar]{Mikhail Katliar$^{1}$\thanks{E-mail:
cepstr@gmail.com}\\
$^{1}$10-79, Burdeynogo str., Minsk 220140, Belarus}
\begin{document}

%\date{Accepted 1988 December 15. Received 1988 December 14; in original form 1988 October 11}

\pagerange{0--0} \pubyear{0000}

\maketitle

\label{firstpage}

\begin{abstract}
The aim is to measure the shapes of galaxies to reconstruct the gravitational lensing signal in the presence of noise. Information about point spread function is provided by star images.

The method is tested on simulated galaxies from the Mapping Dark Matter (MDM) challenge.
\end{abstract}

\begin{keywords}
Gravitational lensing - Methods: numerical, statistical
\end{keywords}

\section{Introduction}
The problem of shape measurement in optical
imaging data is that galaxy images are convolved with a possibly
varying point spread function (PSF) which must be accurately corrected
for when deducing galaxy shape.

\section[]{A probabilistic approach to shape measurement}
The data consists of simulated galaxy images, paired with star images, which provide information about PSF.

An observation of a galaxy and a star yields two surface
brightness distributions denoted by a vector of pixel
values. This vector can be transformed, yielding an observation vector $\y \in \mathds{R}^N$. The shape of the galaxy may be characterized by its two-component
ellipticity $\e \in \mathds{R}^2$.

The goal is to find a value of ellipticity $\hat{\e}(\y)$ such that the expected ellipticity error is minimized:
\begin{equation}
\hat{\e}(\y) = \underset{\e'}{\operatorname{argmin}}\, \E_{\e \mid \y}\left(\left.\left|\e'-\e\right|^2 \,\right|\, \y \right) = \E_{\e \mid \y}(\e \mid \y).
\end{equation}
where $f_{\e \mid \y}$ is the PDF of galaxy ellipticity $\e$ given observation vector $\y$.

\begin{multline}
\E_{\e \mid \y}(\e \mid \y) = \int\limits_\X \e(\x) f_{\x \mid \y}\left(\x \mid \y \right)\,d\x = \\ = \frac{\int_X \e(\x) f_{\y\mid \x}(\y \mid \x) f_{\x}(\x) \, d\x}{\int_X f_{\y\mid \x}(\y \mid \x') f_{\x}(\x') \, d\x'}
\label{eq:estimator}
\end{multline}
where $\x$ is a full vector of image parameters.

The expression for $f_{\y \mid \x}$ can be constructed later based on a model of signal and noise. This will be done in Section~\ref{sec:Signal and noise}.

Prior distribution of model parameters is given by $f_\x$. The simplest choice is to assume uniform prior, in which case $f_\x$ cancels out in \eqref{eq:estimator}. Another choice is to estimate $f_\x$ from a dataset:
\begin{multline}
f_{\x}(\x) = \int f_{\x \mid \y}(\x \mid \y) f_\y(\y) \,d\y \approx \frac{1}{N} \sum_{i=1}^N f_{\x \mid \y}(\x \mid \y_i) = \\ = \frac{1}{N} \sum_{i=1}^N \frac{f_{\y\mid \x}(\y_i \mid \x) f_{\x}(\x)}{\int_X f_{\y\mid \x}(\y_i \mid \x') f_{\x}(\x') \, d\x'}.
\label{eq:prior}
\end{multline}
It seems there is no obvious way to solve \eqref{eq:prior} for $f_\x$. One of the options is to use iterative procedure:
\begin{eqnarray}
f_{\x}^{(0)}(\x) & =& \mathrm{const},\\
f_\x^{(k)}(\x) &=& \frac{1}{N} \sum_{i=1}^N \frac{f_{\y\mid \x}(\y_i \mid \x) f_\x^{(k-1)}(\x)}{\int_X f_{\y\mid \x}(\y_i \mid \x') f_\x^{(k-1)}(\x')\, d\x'}.
\end{eqnarray}

\section[]{Signal and noise model}
\label{sec:Signal and noise}
Let's assume that both intrinsic 2D intensity distribution produced by a galaxy and a PSF have elliptic contour lines:
\begin{equation}
I_g(\rr) = I_{g0} \,p_g\left(\sqrt{\rr^T \A(\s_g)\, \rr}\right)
\end{equation}
\begin{equation}
h(\rr) = p_s\left(\sqrt{\rr^T \A(\s_s)\, \rr}\right)
\end{equation}
where $I_{g0}$ is galaxy intensity factor, $\s_g$ and $\s_s$ are shape parameters of galaxy and PSF, respectively, $\A(\s)=\T(\s)^T\T(\s)$, $\T(\s)$ is a transform from screen coordinates to ``unsheared'' coordinate system, in which intensity contour lines are circular, $p_g$ is galaxy intensity profile, $p_s$ is PSF profile, and $\rr$ is 2D pixel coordinate.

Observed intensity distribution can be defined by expression
\begin{multline}
s = (I_g \otimes h \otimes \Pi_{1\times 1} + n) \Pi_{M\times M} \approx \\ \approx I_g \otimes h \otimes \Pi_{1\times 1} + n\, \Pi_{M\times M},
\label{eq:signal}
\end{multline}
where $\Pi_{x\times y}$ is 2D rectangle function of size $x\times y$, $n$ is a noise. Assuming that $n$ is a Poissonian noise, and noise samples for every pixel are independent, it is possible to draw an expression for likelihood $f_{\y\mid\x}$. However, it is difficult to obtain closed-form expression for convolution in \eqref{eq:signal}, and it is expensive to compute it numerically. It is more convenient to move to frequency domain:
\begin{multline}
\F(\Delta_{1,1} s) \approx \F\left(I_g \otimes h \otimes \Pi_{1\times 1}\right) \otimes \Delta_{1,1} + \F(n \, \Pi_{M\times M}\Delta_{1 \times 1}) =\\= (\F(I_g) \F(h) \F(\Pi_{1\times 1})) \otimes \Delta_{1,1} +\F(n \, \Pi_{M\times M}\Delta_{1,1}),
\label{eq:signal_F}
\end{multline}

If components of the observation vector are statistically independent, then
$$
f_{\y\mid \x}(\y\mid \x) = \prod_{i=1}^N f_{\y \mid \mathbf{\xi},\x}(y_i \mid \mathbf{\xi}_i,\x),
$$
where $\mathbf{\xi}_i$ is a vector of parameters specific to $i$-th component of the observation vector. For example, $\mathbf{\xi}_i$ is a coordinate of $i$th pixel in case if $\y$ is a vector of pixel intensities, and $i$th spatial frequency if $\y$ is an image spectrum.

\begin{multline}
L_{\y\mid \x}(\y\mid \x) = \ln f_{\y\mid \x}(\y\mid \x) = \\ = \sum_{i=1}^N \ln f_{\y \mid \mathbf{\xi},\x}(y_i \mid \mathbf{\xi}_i,\x).
\end{multline}

Let $m\left(\xi,e,\theta\right)$ be an expected value of signal at point $\xi$. Assuming Gaussian noise with variance $\sigma^2$, we get
$$
f_{y \mid \xi,e,\theta}(y \mid \xi_i,e,\theta) = \frac{1}{\sqrt{2\pi} \sigma} \exp\left(-\frac{\left(y-m\left(\xi,e,\theta\right)\right)^2}{2\sigma^2}\right).
$$

$$
m(\xi,e,\theta) = p_s\left(\xi^T A\left(e\right) \xi, \beta\right) p_g\left(\xi^T A\left(e_s\right) \xi\right)
$$

If the sample of galaxies
has a probability distribution of intrinsic ellipticities (i.e. the value
of ellipticity that would be measured by the observer in the absence
of degradation by the PSF or by noise) is $f(e)$, then the probability

\section{Likelihood approximation}
Denote a log likelihood function
\begin{equation}
L_{\y\mid \x}(\y\mid \x) = \ln f_{\y\mid \x}(\y\mid \x).
\end{equation}

Let $L_{\y\mid \x}(\y_i \mid \x)$ reaches its maximum at $\x=\x_i$. As first derivatives vanish at extremum, second-order Taylor approximation to $L_{\y\mid \x}(\y_i \mid \x)$ is
\begin{multline}
L_{\y\mid \x}(\y_i \mid \x) \approx L_{\y \mid \x}(\y_i \mid \x_i)\, +\\+ \frac{1}{2} (\x-\x_i)^T \mathbf{H}_i (\x-\x_i),
\end{multline}

$$
\mathbf{H}_i = D^2_{\x} L_{\y \mid \x}(\y_i \mid \x) \mid_{\x=\x_i}.
$$

\begin{multline}
f_{\y\mid \x}(\y_i \mid \x) = \exp\left(L_{\y\mid \x}(\y_i \mid \x)\right) \approx \\ \approx f_{\y \mid \x}(\y_i \mid \x_i) \exp\left(\frac{1}{2} \left(\x-\x_i\right)^T \mathbf{H}_i (\x-\x_i)\right).
\label{eq:f(y|x) approx}
\end{multline}

By substituting \eqref{eq:f(y|x) approx} into \eqref{eq:f(x|y)}, we get
\begin{equation}
f_{\x\mid \y}(\x\mid \y_i) \approx \frac{e^{\frac{1}{2} (\x-\x_i)^T \mathbf{H}_i (\x-\x_i)} f_{\x}(\x)}{\int e^{\frac{1}{2} (\x'-\x_i)^T \mathbf{H}_i (\x'-\x_i)} f_{\x}(\x') \,d\x'}.
\end{equation}

\begin{equation}
f_{\x}(\x) = \int f_{\x \mid \y}(\x \mid \y) f_\y(\y) \,d\y \approx \frac{1}{N} \sum_{i=1}^N f_{\x \mid \y}(\x \mid \y_i)
\end{equation}

\begin{eqnarray*}
f_{\x}^{(0)}(\x) & =& \mathrm{const}\\
f_{\x\mid \y}^{(1)}(\x\mid \y_i) & =& (2\pi)^{-\frac{n}{2}} \left|-\mathbf{H}_i\right|^{\frac{1}{2}} \times \\
& \times & e^{\frac{1}{2} (\x-\x_i)^T \mathbf{H}_i(\x-\x_i)}\\
f_{\x}^{(k)}(\x) & = & \frac{1}{N} \sum_{i=1}^N f_{\x \mid \y}^{(k)}(\x \mid \y_i)\\
f_{\x\mid \y}^{(k)}(\x\mid \y_i) & =& \frac{e^{\frac{1}{2} (\x-\x_i)^T \mathbf{H}_i (\x-\x_i)} f_{\x}^{(k-1)}(\x)}{\int e^{\frac{1}{2} (\x'-\x_i)^T \mathbf{H}_i (\x'-\x_i)} f_{\x}^{(k-1)}(\x') \,d\x'}\\
f_{\x\mid \y}^{(2)}(\x\mid \y_i) & =& \frac{\sum_{j=1}^N \left|-\mathbf{H}_j\right|^{\frac{1}{2}} e^{\frac{1}{2} \left((\x-\x_j)^T \mathbf{H}_j(\x-\x_j) + (\x-\x_i)^T \mathbf{H}_i (\x-\x_i)\right)}}{ \sum_{j=1}^N \left|-\mathbf{H}_j\right|^{\frac{1}{2}} \int e^{\frac{1}{2} \left((\x'-\x_j)^T \mathbf{H}_j(\x'-\x_j) + (\x'-\x_i)^T \mathbf{H}_i (\x'-\x_i)\right)} \,d\x'}
\end{eqnarray*}

\section{Results}

\section*{Acknowledgments}

 \begin{thebibliography}{99}
\bibitem[\protect\citeauthoryear{Miller et al.}{2007}]{Miller et el. 2007} Miller, L., Kitching, T. D., Heymans, C., Heavens, A. F. and Van Waerbeke, L. (2007), Bayesian galaxy shape measurement for weak lensing surveys � I. Methodology and a fast-fitting algorithm. Monthly Notices of the Royal Astronomical Society, 382: 315�324. doi: 10.1111/j.1365-2966.2007.12363.x

\bibitem[\protect\citeauthoryear{Kitching et al.}{2008}]{Kitching et al. 2008} Kitching, T. D., Miller, L., Heymans, C. E., Van Waerbeke, L. and Heavens, A. F. (2008), Bayesian galaxy shape measurement for weak lensing surveys � II. Application to simulations. Monthly Notices of the Royal Astronomical Society, 390: 149�167. doi: 10.1111/j.1365-2966.2008.13628.x
\end{thebibliography}

\appendix

\bsp

\label{lastpage}

\end{document}
